\documentclass[a4paper,11pt,twoside]{article}
\usepackage[swedish]{babel}
\usepackage[T1]{fontenc}
\usepackage[utf8]{inputenc}
\usepackage{lmodern}
\usepackage{verbatim}
\usepackage{multicol}
\usepackage{hyperref}
\usepackage{listings}
\usepackage{color,hyperref}
\usepackage{csquotes}
\usepackage{fancyhdr}
\usepackage{courier}
\usepackage{color}
\usepackage{amsmath}
\usepackage{graphicx}
\usepackage{bbm}
\usepackage{amssymb}
\usepackage{wrapfig}

\title{Utbytesstudier Prosam}
\author{
  Christopher Lillthors 911005-3817 \\
  \\
  Kurs: Prosam \\
  Kurskod: DD1390
}
\pagestyle{fancy}
\setlength{\headheight}{54pt}
\fancyfoot[C]{}
\fancyfoot[RO, LE] {\thepage}
\lhead{\textbf{Kungliga Tekniska Högskolan} \\ Civilingenjör datateknik\\ Prosam DD1390}
\rhead{\textbf{Utbytesstudier} \\ \date{\today} \\ \ \\}
\setlength{\parindent}{0in}
\setlength{\parskip}{0.1in}
\date{\today}


\begin{document}
\maketitle
\tableofcontents
% for the lulz...
\begin{frame}
\null
\vfill
Generated in \LaTeX
\end{frame}
\thispagestyle{empty}.
\newpage
\clearpage
\setcounter{page}{1}


% notes
% Kulturella skillnader
% språkliga skillnader
% kontaker kvar i landet, besöker dem i sommar
% helt andra värderingar i sånt som vi svenskar kanske inte värdesätter lika högt.
% attraktivt CV.
% man lär sig mycket bara av att vara där. Att uppleva kulturen på riktigt är givande.
% Lokala evenemang och leva in i kulturen. Lärde mig nationalsången till 4 juli...
% försöka engagera sig i lokala grupper och få ut så mycket som möjligt av sin vistelse.


% Skriv en reflekterande text på 600-800 ord där du reflekterar över reseberättelserna, värdet av utbytesstudier (för utbytesstudenten själv och andra), ditt eget intresse för utbytesstudier och arbete utomlands. Du kan finna inspiration i frågorna för årskurs 1 ovan. Reflektera också över hur du studerat i kurserna i period 1 och 2 och om du förändrat din studieteknik något sedan i våras.
% Tänk på att reflektionen i vissa delar måste nå upp till nivå 3-4 för att VG ska kunna ges. Läs igenom beskrivningen av reflektionsnivåerna innan du börjar skriva! Ta gärna ut svängarna i reflektionen och visa att du verkligen har åsikter och motiveringar till åsikterna.


\section{Rapport}
Jag tror att utbyte är en väldigt bra idé och jag skulle själv vilja åka iväg på det. Har sedan tidigare åkt på språkresa till USA, vilket är helt klart den bästa resan jag någonsin gjort. Genom att åka iväg utvecklade jag inte enbart mina språkliga kunskaper utan även lärde mig mycket om hur man lever och bor i USA, även kulturera skillnader


Jag tror att utbytesstudier är ett utmärkt tillfälle att utvecklas som människa och samtidigt ha det roligt. Att åka utomlands och uppleva en annan kultur ger oss andra perspektiv och det är enligt mig mycket utvecklade och givande.
Har själv ingen erfarenhet av just utbytesstudier men har åkt iväg på språkresa till USA i lite drygt 1 månad. Väl där gick jag i skola varje vardag och läste både engelska och vi lärde oss mer om amerikansk kultur. Jag bodde tillsammans med en amerikansk familj, vilka jag fortfarande har kontakt med.

Tycker Viktor Holmbergs berättelse var mest givande och mest beskrivande av dem alla.

\end{document}